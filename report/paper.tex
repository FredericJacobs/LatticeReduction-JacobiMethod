\documentclass[10pt]{article}

\usepackage{amsfonts}
\usepackage{amsmath}

\title{Comparing the Jacobi Method and LLL lattice reduction algorithms for cryptographic applications}
\date{Survey article\\ Fall 2014}
\author{Frederic Jacobs\\ EPFL Bachelor Semester Project\\ frederic.jacobs@epfl.ch}

\begin{document}

\maketitle

\begin{abstract}
In this \cite{originalJacobiMethodLatticeBasisReduction}
\end{abstract}

\section{Introduction}


\subsection{Results and Contributions}

We show the following main results:
\begin{itemize}
\item An analysis of running times depending on factors
\end{itemize}


\subsection{Related Work}

\section{Preliminaries}

In this section, we briefly cover the basics of lattice theory and reduction required in the scope of this article.

\subsection{Lattice}

We will define a lattice both formally an informally. 

Informally\cite{lllAlgorithm}, a lattice can be defined as an infinite arrangement of points in $\mathbb{R}^n$ spaced with sufficient regularity that one can shift any point onto any other point by some symmetry of the arrangement.

This can be translated into a more formal definition\cite{SchnorrStanfordNotes}.\newline 
Let $b1, b2, ... b_n \in \mathbb{R}^m$ be linearly independent vectors. We call the additive subgroup

\[
L(b_1,b_2,...,b_n):= \displaystyle\sum_{i=1}^{n}b_i \mathbb{Z}
\] of $\mathbb{R}^m$, a lattice with basis $b_1,b_2,...,b_n$.  

The \emph{dimension of the lattice} is $dim(L) := rank(L) := n$. 

We will work exclusively with full dimensional lattices. A lattice $L \subseteq \mathbb{R}^m$ is \emph{full dimensional} when $rank(L)=m$. In other words, this means that we will work on matrices with $m$ linearly independent vectors.

\subsubsection{Equivalent basis}

As this article focuses on lattice basis reduction, it is important to define the equivalence of two bases. Two bases $B_1 , B_2$ are equivalent iff $B_2 = B_1 U$ for a unimodular matrix $U$.

A \emph{unimodular matrix} $M$ is a square integer matrix such that $det(M) = \pm 1$. The right multiplication of a matrix $B$ with a unimodular matrix $M$ results can result in the following elementary column operations:
\begin{itemize}
\item Swap two columns
\item Negate the elements of a column
\item Addition of an integer multiple of one column to another
\end{itemize}


\subsubsection{Volume}

The volume of a lattice is defined\cite{SchnorrStanfordNotes} as:
\[
\det L = \text{vol}_n \left. \Big( \displaystyle\sum_{i=1}^{n} t_i b_i \right| 0 \leq t_1, t_2 ..., t_n < 1 \Big)
\]

For the specific subset of lattices that we are studying, full dimensional lattices, the volume is equal to the determinant.

\subsection{Quality of a reduction}

In this section, we will discuss two lattice reduction quality indicators that we will use to compare reductions.

\subsubsection{Orthogonality defect}
The orthogonality defect of a basis $b_1, b_2,...,b_n$ of a lattice $L$ is given by
\[
    \text{OrthDefect}(L) := \frac{\displaystyle\prod^{n}_{i=1} \|b_i\| }{\det(L)}
\]

The orthogonality defect is a quality indicator that we will use to compare the reduction of \emph{the set of basis vectors}.

\subsubsection{Hermite Factor}
The Hermite Factor, sometimes called Hermite constant, of a basis $b_1, b_2,...,b_n$ of a lattice $L$ is given by

\[
    \text{HF}(L) := \frac{\|a_1\|}{(\det(L))^{1/n}}
\]

The Hermite Factor is a quality indicator that we will use to compare the reduction of the \emph{shortest-vector}.

\subsection{Special forms of Lattices}

\subsubsection{Hermite Normal Form}

The Hermite Normal Form representation of a matrix $M$ is unique. It is the analogue of the reduced echelon form for integer matrices.

\subsubsection{Random Lattice}

TODO

\subsection{LLL}
up to 1 page
\begin{itemize}
\item Criteria for reduction
\item Basic steps on algorithm: exchange, reduction
\end{itemize}

\section{Our Fast Jacobi Method Implementation}

\section{Conclusions}

\bibliographystyle{alpha}
\bibliography{references}

\section{Credits}


\end{document}
